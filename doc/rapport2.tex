\documentclass[11pt]{article}
% \documentclass[11pt,twocolumn]{article}

\usepackage[utf8]{inputenc}
\usepackage{amsmath}
\usepackage{mathtools}
\usepackage{amsfonts}
\usepackage{graphicx}
\usepackage{cite}
\usepackage{url}
\usepackage{color}
\usepackage{float}
\usepackage{hyperref}
\usepackage{relsize}
% \usepackage{xspace}

\usepackage{tabularx}

\usepackage{algorithm}
\usepackage{algorithmicx}
\usepackage{algpseudocode}


% \overfullrule=2cm

\newcommand{\husk}[1]{\color{red} #1 \color{black}}
\newcommand{\expect}[1]{\langle{#1}\rangle}

\newcommand{\CC}{C\nolinebreak\hspace{-.05em}\raisebox{.4ex}{\tiny\bf +}\nolinebreak\hspace{-.10em}\raisebox{.4ex}{\tiny\bf +}}
\def\CC{{C\nolinebreak[4]\hspace{-.05em}\raisebox{.4ex}{\tiny\bf ++}}}

\title{Project 2 in FYS4411: Computational Physics 2}
\author{Mathias M. Vege}

\date{\today}
\begin{document}
\maketitle

\begin{abstract}
NAN
\end{abstract}

\tableofcontents

\section{Introduction}
The goal of this project is to study closed shell systems of electrons confined in a harmonic oscillator potential - a quantum dot. Within this scope we are investigating the ground state energies, \husk{exception values kinetic and potential energies, single particle densities and one-body densities}. The system we are interested in is a two dimensional system of $N$ electrons, and since we have closed shell systems we will look at $N=2,6$ and $12$ electrons.



\section{Theory}
As tradition demands we begin by looking at the Hamiltonian of the system we are to solve,
\begin{align}
	\hat{H} = \sum^N_{i=1} \left( -\frac{1}{2}\nabla^2_i + \frac{1}{2}\omega^2r^2_i \right) + \sum^N_{i<j} \frac{1}{r_{ij}}
	\label{eq:hamiltonian}
\end{align}
In order to accommodate a modern notational benefits and simplifications, we use natural units($\hbar = c = e = m_e = 1$). We can also observe that $N$ is the number of particles we are using, and the $\omega$ is the oscillator frequency for the harmonic oscillator part. The first part, we recognize as the unperturbed part of the Hamiltonian,
\begin{align}
	\hat{H}_0 = \sum^N_{i=1} \left( -\frac{1}{2}\nabla^2_i + \frac{1}{2}\omega^2r^2_i \right),
	\label{eq:hamiltonian_unperturbed}
\end{align}
and the last part is the perturbation to our system,
\begin{align}
	\hat{H}_1 = \sum^N_{i<j} \frac{1}{r_{ij}}
	\label{eq:hamiltonian_perturbed}
\end{align}
such that $\hat{H} = \hat{H}_0 + \hat{H}_1$. The distance between two electrons is defined as following,
\begin{align}
	r_{ij} = |\mathbf{r}_i - \mathbf{r}_j| = \sqrt{(x_i - x_j)^2 + (y_i - y_j)^2}
	\label{eq:electron_distance}
\end{align}



\subsection{Variational Monte Carlo}

In order to make any progress with variational Monte Carlo, we need to get ourselves a wave function.



\subsection{Electron wave function}
Our wave function will consist of two parts: one that comes from the harmonic oscillator potential and is built up based on the fermionic nature of the system, and one that proved a correlation between two particles - the so-called Jastrow factor. The wave function we construct, will be called our \textit{trial wave function}.
\begin{align}
	\psi_T(\mathbf{r}) = \psi_C(\mathbf{r})\psi_{OB}(\mathbf{r})
	\label{eq:WF_trial}
\end{align}

\subsubsection{Slater determinants}
The wave function for an electron in a two dimensional harmonic oscillator potential can be written as a Hermite polynomial,
\begin{align}
	\phi_{n_x,n_y}(x,y) = AH_{n_x} (\sqrt{\omega\alpha}x) H_{n_y} (\sqrt{\omega\alpha} y) \exp{\left(-\frac{\omega\alpha}{2}\left( x^2 + y^2 \right)\right)}
\end{align}
The details surrounding the mysterious variational parameter $\alpha$ will be explained further on. For now, we will place these $\phi$'s into a Slater determinant. The Slater determinant is a creature that describes the wave function of a fermionic system, while also enforcing anti-symmetrization and thus the Pauli principle. Our Slater determinant will take the following form, when we describe the specific state of a system by $n_x, n_y$ and a specific particle $\mathbf{r}_i = x_i\mathbf{i} + y_i\mathbf{j}$,
\begin{align}
	Det(\Phi(\mathbf{r})) \equiv \frac{1}{\sqrt{N!}}
	\begin{vmatrix}
		\phi_1(\mathbf{r}_1)	& \phi_2(\mathbf{r}_1) 	& \hdots 	& \phi_N(\mathbf{r}_1) 	\\
		\phi_1(\mathbf{r}_2) 	& \phi_2(\mathbf{r}_2) 	& \hdots 	& \phi_N(\mathbf{r}_2) 	\\
		\vdots 					& \vdots				& \ddots 	& \vdots 				\\
		\phi_1(\mathbf{r}_N) 	& \phi_2(\mathbf{r}_N) 	& \hdots 	& \phi_N(\mathbf{r}_N) 	\\
	\end{vmatrix}
	\label{eq:slater_determinant}
\end{align}
Note that we have defined $\mathbf{r} \equiv (\mathbf{r_1},\mathbf{r_2},\dots,\mathbf{r_N})$. To pull this definition back to our trial wave function \eqref{eq:WF_trial}, we get
\begin{align}
	\psi_{OB}(\mathbf{r}) = Det(\Phi(\mathbf{r}))
	\label{eq:WF_onebody}
\end{align}
Where the $OB$ stands for one body, as in one body wave function. We now need to look into the part that accounts for many body effects, the Jastrow factor.

\subsubsection{The Jastrow factor}
The correlation term is called a \textit{Jastrow factor} is as mentioned here to take into account many-body effects of our system. The general shape of it is
\begin{align}
	\psi_C(\mathbf{r}) = \prod_{i<j}^N \exp{\left(\frac{ar_{ij}}{1 + \beta r_{ij}}\right)} = \prod_{i=1}^N \prod_{j=i+1}^N \exp{\left(\frac{ar_{ij}}{1 + \beta r_{ij}}\right)}
	\label{eq:WF_jastrow}
\end{align}
where the $C$ stands for correlation. The $a$ is a parameter that is $1$ for parallel spin, and $1/3$ for anti-parallel spin. The $\beta$ is another variational parameter and the $r_{ij}$ is defined by the equation \eqref{eq:electron_distance} as the distance between two electrons. For now, we shall begin by looking closer at the two-electron case.




\subsection{Two electron case}
For the two electron case the Hamiltonian takes on familiar form,
\begin{align}
	\hat{H} = -\frac{1}{2}\left(\frac{\partial^2}{\partial x^2_1} + \frac{\partial^2}{\partial y^2_1} + \frac{\partial^2}{\partial x^2_2} + \frac{\partial^2}{\partial y^2_2}\right) + \frac{1}{2}\omega^2(x^2_1 + y^2_1 + x^2_2 + y^2_2) + \frac{1}{r_{12}}
	\label{eq:hamiltonian_two_electron}
\end{align}

\subsubsection{Unperturbed local energy}

\subsubsection{Local energy}

\subsubsection{Quantum force}



\subsection{\texorpdfstring{$N$}{N} electron case}

\subsubsection{Unperturbed local energy}

\subsubsection{Local energy}

\subsubsection{Quantum force}




\subsection{The Metropolis Algorithm}

\subsubsection{Steepest descent}

\subsubsection{Importance sampling}



\section{Implementation}



\section{Results}



\section{Discussion and conclusion}


\husk{REFERANSER!!}

\end{document}